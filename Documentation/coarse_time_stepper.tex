

To simulate the time evolution of  the density $\U(t)$, we construct  a coarse time stepper $\cts$ which allows the performance of time-steps at the macroscopic level, using only the stochastic simulation of the position vectors of the $N$ particles at the microscopic level, %$\mathbf{X}(t+ n \Delta t)$, 
generated by eq. \eqref{Euler-Mar}. 

To achieve this, we will define two operators (lifting in subsection \ref{section:lifting} and
restriction in subsection \ref{section:restriction}) that relate the microscopic and macroscopic levels of description.
Once these lifting $\mathcal{L}$ and restriction  operators $\mathcal{R}$ have been constructed, a coarse time-stepper $\cts$ 
to evolve the macroscopic state $\U$ over a time interval of length $n \Delta t$ is constructed as a three-step-procedure (lift–evolve–restrict):

\begin{equation}
\U(t + n \Delta t) = \cts(\U) = (\mathcal{R} \circ \mathcal{E}(n \Delta t) \circ  \mathcal{L}(\mathbf{\omega})  ) (\U(t))
\end{equation}
where $\mathcal{E}(n \Delta t)(\U(t))$ is the simulation of the SDE for $N$ particles over $n$ timesteps.


\subsubsection{Lifting: $\U \rightarrow \mathbf{X} $} \label{section:lifting}
Given the density \U, we need to sample a position vector $X_i$ for every particle $i \leq N$. We compute $X$ from a $N$-dimensional vector $\bf{U}$  with uniform random elements $U_i \in [0,1]$ such that $\U(X_i)=U_i$, using the inverse transformation method. The particle does not only gets an inital position, but also a seed for generating random steps in the simulation.

\subsubsection{Restriction:  $\mathbf{X} \rightarrow \U $}
\label{section:restriction}
The restriction operator $\mathcal{R}: \mathbb{Q}^N \rightarrow \mathbb{Q}^k$ maps the microscopic state $\mathbf{X}$ (determined by the position vectors of $N$ particles) to a density \U, discretisized in $k$ bins. This is done by counting the number of particles in every bin $\Delta_j$ for $1\leq j \leq k$: %= x_{j+\frac{1}{2}}-x_{j-\frac{1}{2}}$. 

\begin{equation}
\frac{1}{N} \sum_{i=1}^{N}  w^i \cdot \chi_{\Delta_j}(X^i) = \rho_j  \label{restriction}
\end{equation}
with 
\begin{equation}
\chi_{\Delta_j}(X) = \begin{cases}
  1 & \mbox{if } X \in \Delta_j, \\
  0 & \mbox{if } X \notin \Delta_j. \\
\end{cases}
\end{equation}
%\begin{equation}
% \frac{1}{k}\sum_{j=1}^{k} \rho_j =  \frac{1}{N} \sum_{i=1}^{N}  w_i x_i
%\end{equation}
and setting all weights $w_i =1 $ for $1\leq i \leq N$.

The reason why we explicitly introduced these weights in the restriction operator will be clarified in section \ref{Jv_approx} where we will need to evaluate  the coarse time stepper  $\cts (\U + \varepsilon \mathbf{v})$, now applied to the density shifted with a certain perturbation $\varepsilon \mathbf{v}$. %We explicitly introduced these weights in the restriction operator \eqref{restriction} because they can help to reduce the variance in the jacobian-vector-products. In the finite difference approximation we need to evaluate $\cts (\U + \varepsilon \mathbf{v})$, the coarse time stepper applied to the density shifted with a certain perturbation.
%For the weighted restriction operator, we choose different weights $w_i$ for every particle. (we only use this for calculating the jacobian-vector-produ cts. (Why? Well, the weights are just one if we dont use a perturbation)f
To evaluate the perturbated restriction-operator we will use the weights $ w^i_{\varepsilon}$, determined such that
\begin{equation}
\frac{1}{N} \sum_{i=1}^{N}  w^i_{\varepsilon} \cdot \chi_{\Delta_j} (X^i) = \rho_j + \varepsilon v_j . \label{restriction_eps}
\end{equation}
We do this by computing the weight per bin as $ w^j_{\varepsilon} = 1+ \frac{\varepsilon v_j}{\rho_j}$ and assign this value to each particle in $\Delta_j$. So, small perturbations on the density lead to small perturbations in the weights. The advantage of this weighted restriction operator lies in the possibility to use the same realizations $\mathbf{X}$ in the unperturbated \eqref{restriction} and the perturbated  \eqref{restriction_eps} restriction-operator.
